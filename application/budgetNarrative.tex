\section*{Budget Narrative}

\textsl{The budget component of your grant proposal consists of the
  Short form budget template and the corresponding budget
  narrative. Together, the template and narrative provide a clear
  picture of the financial resources you are requesting to support the
  project. The budget template is contained in a separate Excel
  document that should be submitted with your proposal. Instructions
  for completing the budget template and definitions of relevant terms
  are included in the template document.}

\paragraph{Personnel and Benefits} 
\textsl{What are the roles and responsibilities of each full-time
  equivalent (FTE) or group of FTEs supporting the project, the
  rationale for the number of requested FTEs, and the methodology for
  estimating the base salaries (e.g., actual salaries or estimates
  based on similar job descriptions)?}

\vspace{\topsep}

We will use \$5000 to hire TA's at hourly rates to monitor the
forums and seek-out and find potential problems before they become
problems for students. These TA's in turn will then be experts at
using our system and will be instrumental presenters at the
\mooculus\ conferences.


\paragraph{Consulting and Professional Fees}
\textsl{Include a brief description of the work to be performed in
  support of the overall project, the current status of the
  contract(s) (e.g., confirmed or projected), and the cost assumptions
  used (including estimates of the number of days to be worked and the
  daily rate).}

We already have the \mooculus\ platform, but there is significant room
for growth, particularly with regards to the ``adaptive'' features of
the platform; currently the platform's hidden Markov model uses
transition probabilities that were simply guessed from an instructor
answering some sample questions---the parameters of the model ought to
be discovered from the data, say by the Baum-Welch algorithm.
Additionally, the \mooculus\ platform does not allow any follow-up
questions to be asked, and questions are not ranked in any way by
difficulty, so even with the hidden Markov model, we cannot currently
adaptively funnel students into more difficult or less difficult
questions based on performance.

These improvements can be made with about six person-weeks of work at
\$75/hour and would be negotiated by contracting with \gratisu; the
work that \gratisu\ performs would be open-source, so the platform
could be widely used by the educational community.

For evaluation, we will partner with West Monroe Partners, LLC; we
expect about two person-weeks of work from the evaluation team, billed
at \$50/hour.

\paragraph{Travel and Accommodations}
\textsl{Include a brief description of the travel required for this
  project.  Include the methodology used to calculate the total cost
  estimates for each trip, the assumptions used to determine the
  appropriate number of trips, and the rationale for how those trips
  will support achievement of the results of the project.}

\vspace{\topsep}

Holowinsky and Snapp will travel to the AMS/MAA Joint Meetings during
January 15--18, 2014, to host a booth advertising this project.
Hosting a booth costs approximately \$1000; travel and lodging for
will be around \$2000.


\paragraph{Conferences, Conventions, and Meetings}
\textsl{Include a brief description of the meetings required for this
  project. Include the methodology used to calculate the total cost
  estimates for each meeting, including the estimated number of
  attendees and total cost per attendee. What is the rationale for how
  those meetings will support achievement of the results of the
  project?}

\vspace{\topsep}

During the spring semester we will host the First
\mooculus\ Conference. This will be a small conference of around 20
people for 5 days. We will send out invitations to a variety of
schools, from peer institutions, to smaller liberal arts
schools. Estimating lodging at \$100 per day and airfare at \$300 per
participant, we arrive at a cost of \$16000. During this conference,
we will share our experience with \mooculus, and discuss how this
technology could be used at their institutions.

During the summer we will host the First \mooculus\ Workshop. The
logistic details of this conference are similar to those of the first
conference---with two changes. First we will invite participants of
the first conference who utilized \mooculus\ in their classrooms to
share their experiences. Second, we will invite an additional 20 local
high school calculus teachers. Since these are local participants,
this should not raise the overall cost of the conference. Again, we
will share our experience with \mooculus, and give strategies as to
how it could be use in class. In addition, we will gather feedback for
making it more useful to other calculus instructors.
