\section{Proposal Narrative}

% Proposal Narrative covering the following topics:

\begin{question}
Please describe the adaptive course(s) to be implemented (max 750
words). We recognize that the course that you have in mind may
already exist but may also be significantly modified or altered in
order to be delivered in a more adaptive mode. The narrative should
provide a detailed description of 
\begin{enumerate}[i.]
\item how you intend to deploy adaptive technologies and methodologies
  in the proposed course(s),
\item the course design, pedagogical or learning model that informs
  the course(s),
\item the specific role of instructors and faculty in delivering
  instruction and designing the course(s),
\item the way in which you will assess student learning and mastery in
  the course(s),
\item an explanation of how the adaptive course meet or exceeds
  mastery requirements and learning outcomes for a similar
  non-adaptive course taught at your institution,
\item the specific advantages or benefits of partnering with the
  selected vendor/adaptive learning solution to deliver a high quality
  learning experience for students, and,
\item describe the level of integration between the adaptive learning
  provider and on-campus learning infrastructure necessary deliver the
  course(s).
\end{enumerate}
\end{question}

% (i) how you intend to deploy adaptive technologies and methodologies
% in the proposed course(s), 

Currently, \mooculus\ is a platform designed for delivering interactive
content to students. In the spring of 2013, we used this platform to
deliver a massive open online calculus course by the same name. The
platform and course material are hosted on Ohio State University
servers and is maintained by a small team of researchers. We will
enhance \mooculus\ with adaptive technology, allowing it to serve as a
supplement for both teaching and learning.

% (ii) the course design, pedagogical or learning model that informs
% the course(s), 

The course will be broken down by topic, and each topic will be recast
as a mini-course. Students wishing to learn a specific topic will be
assessed with a set of diagnostic exercises that will then determine a
curriculum for the student within this mini-course. The curriculum
will consist of online videos, adpative exercises, and selected
textbook readings. This model will facilitate both self-study on the
student's part, and allow teachers to easily integrate our materials
into their courses.

% (iii) the specific role of instructors and faculty in delivering
% instruction and designing the course(s),

Faculty at OSU will produce videos, and online assessment
problems. They will work with \gratisu\ to enhance the flexibility of
their platform, helping them produce a set of tools that will be
adaptable to a variety of different courses.


% (iv) the way in which you will assess student learning and mastery
% in the course(s), 

To assess student learning, we will use a hidden Markov model. Our
experience with \mooculus\ has taught us that an unprecessedented
amount of data can be obtained though online assessment. The hidden
Markov model will be able to utilize nearly all of this data to help
us assess student learning.


% (v) an explanation of how the adaptive course meet or exceeds
% mastery requirements and learning outcomes for a similar
% non-adaptive course taught at your institution, 

We envision assignements without ``hard'' deadlines. The instructor
may suggest that a given assignment is completed at a certain time,
but in fact the students will be able to complete the assignments at
any given time. As a student works each exercise, their performance
will determine how their personal study will proceed. Strong
performance will lead to deeper and more difficult questions. Weak
performance, will be strengthened by directing the student back to the
basics, and then, having mastered those concepts, to the original
goal.


% (vi) the specific advantages or benefits of partnering with the
% selected vendor/adaptive learning solution to deliver a high quality
% learning experience for students, and,

\gratisu, based in Columbus Ohio, has proved to be resposive and
completetly open in regards to development. This means that if a
faculty memeber has an idea, this idea can be prototyped and deployed
in a matter of days. Moreover, the open nature of the tools provided
by \gratisu\ allows the researchers at OSU to study their own pedagogy. We believe this represents a transformative era in mathematics education. 

% (vii) describe the level of integration between the adaptive
% learning provider and on-campus learning infrastructure necessary
% deliver the course(s).


We envision bi-weekly meetings with \gratisu\ discussing how the
current courses are working, current options for the upcoming
curriculum, and enhancements for the future. \gratisu\ will work with
OSU's \textsl{Digital First} initivative, and peliminary successes
will be presented to the public at large via OSU's \textsl{STEAM
  Factory} outreach program. In addition, through \textsl{Math Circles
  for Teachers} we will involve local high school calculus
teachers---they too will be able to make us of these materials.  In
particular, we want \mooculus\ to be
\begin{itemize}
\item A resource for students who are enrolled in an (unrelated)
  traditional first calculus course.
\item A resource for teachers of a traditional calculus course wishing
  to ``flip'' their classroom.
\item A resource for teachers of a traditional calculus course wishing to assign
online skill practice to their students.
\end{itemize}
Adaptive technology is critical to the success of these goals---as a
wide variety of students will be using this tool. REWRITE!!!







\begin{question}
Please describe the faculty, instructors, instructional
designers, or organization(s) involved in the selection, design or
implementation of the adaptive course(s) (max 200 words). Please
provide relevant background information on how long the organization
has been in operation, its mission and market penetration and any
products that are available. Please list the team members, specify
the activities each person will be involved in, and include any
recognition for leadership in your field. Please tell us why you
have confidence that this is the right organization or team to carry
this work forward and achieve implementation success in this grant
program.
\end{question}

Roman Holowinsky has been a professor in the OSU Math Department since
Fall 2010. He is an Alfred P. Sloan fellow and the recipient of the
2011 SASTRA Ramanujan prize. Holowinsky has had an interest in
technology-enhanced courses, and took the lead in forwarding OSU's
Steam Factory public outreach initiative.

Jim Fowler...

Bart Snapp is an expert in the education of future educators and has
been involved with technology in the classroom and distance education
since 1997.

Steve Gubkin/Johnson... 

Johann Thiel...

Corey...




\begin{question}
Short narrative describing a high level work plan with budget
that details both the course design and implementation process,
including the role played by the adaptive technology solution and
provider (max 100 words).
\end{question}


\begin{question}
Please describe how adaptive learning aligns with your
institution’s strategic plan, and if shown to be successful in the
initial implementation, how will the use of such advanced learning
solutions be adopted or scaled within your institution (max 100
words).
\end{question}


\begin{question}
Applicants are also requested to provide a 1 page description
(with budget estimates) of their evaluation approach and the
detailed metrics that they will monitor and track over the course of
the project’s lifecycle and during the implementation of these
courses. The foundation will require all RFP winners to participate
in evaluation and expects its grantees to plan and work diligently
to evaluate the efficacy of the selected projects in improving
intended outcomes related to student success in the course and
mastery. The foundation expects winners to participate actively with
the field as part of the program’s learning community, sharing their
implementation strategies and results as these emerge. Applicants
are requested to describe their capacity to collect, analyze, and
share data with others outside the project, for project evaluation
and in support of secondary research projects or more broader
dissemination. Please list any limitations you have for sharing data
and research.
\end{question}

For evaluation, we plan to utilize the expertise of West Monroe
Partners, LLC. The current plan is for the evaluation to be a three-step process:
\paragraph{Evaluation Development} The consultant and project
  manager will partner with the OSU Math Department to develop the
  evaluation.  This activity will take place at the OSU campus.
\paragraph{Usability Assessment} The consultant will conduct the usability
  assessment to complete the evaluation that was developed in the
  first phase.  The project manager will be responsible for reviewing
  the findings and validating any issues discovered in the assessment.
  This assessment will include identified planning and metrics to
  assess the performance of students utilizing the course versus a
  live course.
\paragraph{Preparation of Final Report} Per the grant guidelines, the
  consultant and project manager will develop the final report and
  conform to the format required by OSU.

