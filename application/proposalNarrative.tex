\section{Proposal Narrative}

% Proposal Narrative covering the following topics:

\begin{question}
Please describe the adaptive course(s) to be implemented (max 750
words). We recognize that the course that you have in mind may
already exist but may also be significantly modified or altered in
order to be delivered in a more adaptive mode.
\end{question}

\begin{subquestion} 
how you intend to deploy adaptive technologies and methodologies in
the proposed course(s)
\end{subquestion}

We propose adapting \mooculus\ for a course entitled ``Calculus for
Middle Grades Teachers.''  \mooculus\ is a platform designed for
delivering interactive math content to students; having developed this
platform during Fall~2012, we deployed the platform during Spring~2013
to deliver a massive open online calculus course on Coursera, called
simply ``Calculus One.''  The platform and course material are hosted
on Ohio State University servers at \url{https://mooculus.osu.edu/}
and is maintained by a small team of researchers.  

The student population for ``Calculus for Middle Grades Teachers''
begins the course with a wide variety of backgrounds; by running an
adaptive course, we can address gaps in future teachers' backgrounds,
while at the same time modeling interactive, engaging educational
experiences for future teachers.

The trouble is that \mooculus\ needs further development work, and
there aren't so many software developers inside the mathematics
department; by partnering with coders from \gratisu, we can build a a
superior version of \mooculus, making use of what we learned from our
Coursera course to improve the student experience.

\begin{subquestion}
 the course design, pedagogical or learning model that informs
 the course(s)
\end{subquestion}

The course will be broken down by topic, and each topic will be recast
as a mini-course.  Students wishing to study a specific topic will
first be assessed with a set of diagnostic exercises that will
determine a curriculum for the student within this mini-course.  The
curriculum will consist of online videos, adpative exercises, selected
textbook readings, and reflections in the forum.  This model will
facilitate both self-study on the student's part, and allow teachers
to learn how to use \gratisu's platform for their own courses.

\begin{subquestion}
 the specific role of instructors and faculty in delivering
 instruction and designing the course(s)
\end{subquestion}

Faculty at OSU will produce videos and the online assessment problems.
They will work with \gratisu\ to design an adaptive platform,
providing pedagogical guidance as they develop a math exercise
platform that will be adaptable to a variety of math courses at a
variety of academic institutions.

\begin{subquestion}
 the way in which you will assess student learning and mastery
 in the course(s), 
\end{subquestion}

To assess student learning, we expand on our work with hidden Markov
models that we began with \mooculus. Our experience with \mooculus\
has taught us that an unprecessedented amount of data can be obtained
though online assessment, but it can be hard to interpret this data.
By training the hidden Markov model with the Baum-Welch algorithm, we
will have more robust student assessment tha \mooculus\ provided.

\begin{subquestion}
 an explanation of how the adaptive course meet or exceeds
 mastery requirements and learning outcomes for a similar
 non-adaptive course taught at your institution, 
\end{subquestion}

The usual calculus course is full of hard deadlines which often stymie
student success---if a student takes just a bit too long to master a
topic, a student can find that they are now hopelessly behind.  To
remedy this, our adaptive course features no hard deadlines;
nevertheless, at the end of the course, the students will be evaluated
on the same final exam as we would give in a non-adaptive course.
This not only ensures that our adaptive course exceeds the usual
learning outcomes, but also provides feedback as to the effectiveness
of our adaptive techniques.

We do, however, understand the importance of providing a ``cohort'' of
students who are at the same place in the course.  So we expect the
instructor may suggest that a given assignment is completed at a
certain time, but nevertheless the students will be able to complete
the assignments at any given time. As a student works each exercise,
their performance will determine how their personal study will
proceed.  Strong performance will lead to more difficult questions;
weak performance, will be strengthened by directing the student back
to the basics, and then, having mastered those concepts, to the
original goal.

\begin{subquestion}
 the specific advantages or benefits of partnering with the
 selected vendor/adaptive learning solution to deliver a high quality
 learning experience for students, and,
\end{subquestion}

The mathematics department has experts in math content and math
pedagogy, but does not have nearly so much experience with software
development.  By partnering with a nonprofit organization outside of
the math department, we can catalyze the wider open-source community
to build an web platform for presenting online, interactive math
exercises.  Having the backend code developed by a nonprofit and
available on github makes it easy for other institutions to study and
to improve the platform, and for other institutions to use the
platform for their own courses.

\gratisu\ is based in Columbus, Ohio, and its board includes a members
of the Ohio State mathematics department, and includes Corey Staten,
who did the programming for WExMOOC, another Coursera MOOC from Ohio
State which was funded by a grant from the Bill \& Melinda Gates
Foundation.

\begin{subquestion}
 describe the level of integration between the adaptive
 learning provider and on-campus learning infrastructure necessary
 deliver the course(s).
\end{subquestion}

We envision twice-weekly meetings with a programming team provided by
\gratisu\ to discuss requirements for the platform and to train
faculty on the resulting XML format for building interactive math
exercises.  \gratisu\ will meet with OSU's \textsl{Digital First}
initivative, and present to the public at large via OSU's
\textsl{STEAM Factory} outreach program.  Finally, through
\textsl{Math Circles for Teachers} we will involve local high school
calculus teachers---they too will be able to make us of these
materials.

In particular, OSU~faculty will be building the first courses for the \gratisu\ platform; the result will provide
\begin{itemize}
\item resources for students who are enrolled in an (unrelated) traditional first calculus course at OSU or any other institution;
\item resources for teachers of a traditional calculus course wishing to ``flip'' their classroom by using our materials;
\item resources for teachers of a traditional calculus course wishing to assign online skill practice to their students.
\end{itemize}
By making significant use of the hidden Markov model, these
experiences will keep students in the zone of proximal development.

\begin{question}
  Please describe the faculty, instructors, instructional designers,
  or organization(s) involved in the selection, design or
  implementation of the adaptive course(s) (max 200 words).

  Please provide relevant background information on how long the
  organization has been in operation, its mission and market
  penetration and any products that are available. Please list the
  team members, specify the activities each person will be involved
  in, and include any recognition for leadership in your field. Please
  tell us why you have confidence that this is the right organization
  or team to carry this work forward and achieve implementation
  success in this grant program.
\end{question}

Faculty involvement includes Roman Holowinsky and Bart Snapp, and
indirectly Jim Fowler through the nonprofit organization \gratisu.
\begin{description}
\item[Roman Holowinsky] has been a professor in the OSU Math
  Department since Fall~2010. He is an Alfred P.~Sloan fellow and the
  recipient of the 2011 SASTRA Ramanujan prize.  Holowinsky has had an
  interest in technology-enhanced courses, and took the lead in
  developing OSU's Steam Factory public outreach initiative.  He will
  be assisting with public outreach activities.
\item[Bart Snapp] is an expert in the education of future educators
  and has been involved with technology in the classroom and distance
  education since 1997.  He will be the instructor of record for the
  adaptive calculus course for middle grades teachers.
\item[GratisU] is a newly founded nonprofit entity; its mission is to
  promote and develop free and open learning technology, and includes
  on its board of directors Jim Fowler (a program director in the
  department of mathematics) and Steven Gubkin (a graduate student in
  the department of mathematics), both of whom were involved with
  \mooculus\ and the Coursera Calculus One course.
\end{description}
This team has already demonstrated success in the Coursera Calculus
One course, using the \mooculus\ platform.

%Jim Fowler is a program director in the department of mathematics who
%is building technology and testing it in the classroom.  His projects
%include a clicker system, game show buzzers, augmented reality setup
%superimposing graphics on a lecturer, TeX to iPad application
%converter, exam generator with associated optical mark recognizer.

%, and Jim
%Fowler (who also sits on the board of directors of \gratisu).

\begin{question}
Short narrative describing a high level work plan with budget
that details both the course design and implementation process,
including the role played by the adaptive technology solution and
provider (max 100 words).
\end{question}

During the first year, \gratisu\ build a platform for the initial run
of the adaptive course; in future years, \gratisu\ follows up with
iterative improvements to the platform based on feedback from the
previous years.  OSU hosts two conferences to train instructors at
other institutions to use the \gratisu\ platform, and offers a
mini-course at the Joint Math Meetings to train other mathematicians
to use adaptive techniques in their own courses.

\begin{question}
Please describe how adaptive learning aligns with your
institution’s strategic plan, and if shown to be successful in the
initial implementation, how will the use of such advanced learning
solutions be adopted or scaled within your institution (max 100
words).
\end{question}

The College of Arts and Sciences at Ohio State has an eLearning
committee which has made recommendations for how eLearning will fit
into the College and University's strategic plan.  In particular,
these recommendations suggest that the College offer online open
courses; the course that we're proposing would count as one of those
courses and publish the results.  Our adaptive format is particularly
helpful for lifetime learning.

By involving graduate students, we provide our OSU students with some
online teaching experience, so instead of having graduate students
running recitations, we are training them to develop engaging, online
student experiences.

\begin{question}
Applicants are also requested to provide a 1 page description
(with budget estimates) of their evaluation approach and the
detailed metrics that they will monitor and track over the course of
the project’s life cycle and during the implementation of these
courses. The foundation will require all RFP winners to participate
in evaluation and expects its grantees to plan and work diligently
to evaluate the efficacy of the selected projects in improving
intended outcomes related to student success in the course and
mastery. The foundation expects winners to participate actively with
the field as part of the program’s learning community, sharing their
implementation strategies and results as these emerge. Applicants
are requested to describe their capacity to collect, analyze, and
share data with others outside the project, for project evaluation
and in support of secondary research projects or more broader
dissemination. Please list any limitations you have for sharing data
and research.
\end{question}

When considering evaluation, it is important to emphasize that
\textbf{evaluation should focus on usability.}  There is, now, clear
evidence that mastery learning improves student outcomes in math
courses; the need is not to provide more evidence for this fact---the
deep need is to understand why successful educational methods are not
being widely deployed.  Anecdotally, the issue is that faculty find it
too difficult to use the technology, and students have trouble, say,
inputting their answers.

Consequently, for evaluation, we plan to utilize the expertise of West Monroe
Partners, LLC. The current plan is for the evaluation to be a three-step process:
\paragraph{Evaluation Development} The consultant and project
  manager will partner with the OSU Math Department to develop the
  evaluation tool.  This activity will take place at the OSU campus.  The assessment will focus on student use of the website; we expect that improvements in student outcomes will take many years to see significant results, so focusing on usability provides an immediate and visible change that can be measured by the evaluation team; the main question to address are barriers to wider adoption.
\paragraph{Usability Assessment} The consultant will conduct the usability
  assessment to complete the evaluation that was developed in the
  first phase.  The project manager will be responsible for reviewing
  the findings and validating any issues discovered in the assessment.
  This assessment will include identified planning and metrics to
  assess the performance of students utilizing the course versus a
  live course.  \gratisu\ will make use of the usability assessment to improve the platform.
\paragraph{Preparation of Final Report} Per the grant guidelines, the
  consultant and project manager will develop the final report and
  conform to the format required by~OSU.

\paragraph{Data sharing} We have no limitations on sharing of anonymized data.
